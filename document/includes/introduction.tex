% The introduction goes here. TODO.

Fast development of technology in recent years, especially in computing power
and data storage, has enabled the machine learning field to evolve rapidly.
This led us to the development of versatile ML models, which are used in wide
range of applications. LLMs are one of the most popular types of ML models
nowadays, which has shown great advancements and now has a lot of practical
applications.\newline

The nature of the interaction with LLM's makes us consider them as helpful
assistants in various tasks. The most appreciated feature of LLM's is their
ability to transfer from natural language instructions to executable commands.
\newline

One of common fields, where such feature can be utilized and increase the QoL
of users (from developers to regular OS users, as now Linux distributions are
becoming more user-friendly and popular), is the command line interface. It is
to remain an essential part of modern development (and related stuff) processes
as it enables us to perform tasks quickly and efficiently.\newline

This is what motivates us to create and evaluate a command line assistant,
which leverages LLM to break the barrier between NL instructions and computer
commands, yielding the need to learn specific syntax for each command from the
user.
\newline

The limitations of remote LLMs often present significant challenges in
real-world scenarios. Relying on a network connection introduces latency,
potential downtime, and dependency on external services.  Furthermore, data
privacy and security concerns frequently restrict the use of external LLMs,
especially when dealing with sensitive information or proprietary tools.
Consider the scenario of a developer needing to quickly execute a complex
series of commands for debugging, or a system administrator requiring immediate
assistance in resolving a system issue – a network outage could be
devastating.\newline

Therefore, this thesis aims to evaluate local LLMs as a viable solution for
command-line assistance. By running these models directly on the user's
machine, we eliminate the dependency on external services, reduce latency, and
address concerns regarding data privacy.  This approach allows for a focused
investigation into the performance and usability of LLMs in a low-latency,
secure environment.\newline

The main goal of this thesis is to describe the process of design, development,
and to evaluate a Linux CLI assistant powered by different local LLM models
(differing in architecture, size, and capabilities and under different
constraints) to see how well they perform in various usage scenarios.\newline

