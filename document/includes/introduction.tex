% The introduction goes here. TODO.

Fast development of technology in recent years, especially in computing power and data storage, has enabled the machine learning field to evolve rapidly. This led us to the development of versatile ML models, which are used in wide range of applications. LLMs are one of the most popular types of ML models nowadays, which has shown great advancements and now has a lot of practical applications.\newline

The nature of the interaction with LLM's makes us consider them as helpful assistants in various tasks. The most appreciated feature of LLM's is their ability to transfer from natural language instructions to executable commands.
\newline

One of common fields, where such feature can be utilized and increase the QoL of users (from developers to regular OS users, as now Linux distributions are becoming more user-friendly and popular), is the command line interface. It is to remain an essential part of modern development (and related stuff) processes as it enables us to perform tasks quickly and efficiently.\newline

This is what motivates us to create and evaluate a command line assistant, which would break the barrier between NL instructions and computer commands, yielding the need to learn specific syntax for each command from the user.
\newline

The main goal of this thesis is to describe the process of design, development, and to evaluate a Linux CLI assistant powered by different local LLM models (differing in architecture, size, and capabilities) to see how well they perform in various usage scenarios.
